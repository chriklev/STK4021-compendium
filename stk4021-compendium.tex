\documentclass{article}

\title{STK4021 Applied Bayesian Analysis compendium}
\author{}

\begin{document}
    \maketitle

    \tableofcontents

    \section{Markov chain Monte Carlo}

        \subsection{Gibbs sampler}

            \subsubsection{Determined scan Gibbs sampler}

                \subsubsection*{With two parameters}
                    Gibbs sampling is practical when you wish to sample $\theta_1, \theta_2 \sim p(\theta_1, \theta_2)$, but cannot use:
                    \begin{itemize}
                        \item direct simulation
                        \item accept-reject method
                        \item Metropolis-Hasting
                    \end{itemize}
                    But you can sample from:
                    \begin{itemize}
                        \item $p(\theta_1 | \theta_2)$ and
                        \item $p(\theta_2 | \theta_1)$
                    \end{itemize}

                \subsubsection*{Algorithm}
                    \begin{enumerate}
                        \item Select intial values for the parameters $\theta^{(0)}$
                        \item Repeat for a given number of iterations, or untill some end condition is met:
                        \begin{enumerate}    
                            \item for each subset $\theta_j$ of $\theta$:
                            \begin{enumerate}
                                \item sample from $p\left(\theta_j^{(t)} | \theta_{-j}^{(t-1)}, y\right)$
                            \end{enumerate}
                        \end{enumerate}
                    \end{enumerate}

            \subsubsection{Random scan Gibbs sampler}
                Special case of Metropolis-Hastings.
                randomly select which subset og theta to update each iteration.

\end{document}